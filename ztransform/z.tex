% \iffalse
\let\negmedspace\undefined
\let\negthickspace\undefined
\documentclass[journal,12pt,twocolumn]{IEEEtran}
\usepackage{cite}
\usepackage{amsmath,amssymb,amsfonts,amsthm}
\usepackage{algorithmic}
\usepackage{graphicx}
\usepackage{textcomp}
\usepackage{xcolor}
\usepackage{pgfplots}
\usepackage{txfonts}
\usepackage{listings}
\usepackage{enumitem}
\usepackage{mathtools}
\usepackage{gensymb}
\usepackage{comment}
\usepackage[breaklinks=true]{hyperref}
\usepackage{tkz-euclide} 
\usepackage{listings}
\usepackage{gvv}                                        
\def\inputGnumericTable{}                                 
\usepackage[latin1]{inputenc}                                
\usepackage{color}                                            
\usepackage{array}                                            
\usepackage{longtable}                                       
\usepackage{calc}                                             
\usepackage{multirow}                                         
\usepackage{hhline}                                           
\usepackage{ifthen}                                           
\usepackage{lscape}

\newtheorem{theorem}{Theorem}[section]
\newtheorem{problem}{Problem}
\newtheorem{proposition}{Proposition}[section]
\newtheorem{lemma}{Lemma}[section]
\newtheorem{corollary}[theorem]{Corollary}
\newtheorem{example}{Example}[section]
\newtheorem{definition}[problem]{Definition}
\newcommand{\BEQA}{\begin{eqnarray}}
\newcommand{\EEQA}{\end{eqnarray}}
\newcommand{\define}{\stackrel{\triangle}{=}}
\theoremstyle{remark}
\newtheorem{rem}{Remark}
\begin{document}
\parindent 0px
\bibliographystyle{IEEEtran}
\title{GATE: ME - 14.2022}
\author{EE22BTECH11219 - Rada Sai Sujan$^{}$% <-this % stops a space
}
\maketitle
\newpage
\bigskip
\section*{Appendix}
\begin{enumerate}
    \item
	The $Z$-transform of $x(n)$ is defined as
    \begin{align}
        X(z) = \sum_{n=-\infty}^{\infty}x(n)z^{-n}
    \end{align}
    If,
    \begin{align}
        x\brak{n}&\overset{\mathcal{Z}}{ \longleftrightarrow}X\brak{z}  \\
        y\brak{n}&\overset{\mathcal{Z}}{ \longleftrightarrow}Y\brak{z}
    \end{align}
    The properties of $Z-$transform can be given as:
    \item Linearity Property:    \\
    \begin{align}
        ax\brak{n}+by\brak{n}&\overset{\mathcal{Z}}{ \longleftrightarrow}aX\brak{z}+bY\brak{z}
    \end{align}
    \item Time shifting property:    \\
    \begin{align}
        x\brak{n-k}&\overset{\mathcal{Z}}{ \longleftrightarrow}z^{-k}X\brak{z}
    \end{align}
    \item Time scaling property: \\
    \begin{align}
    x\brak{\frac{n}{k}}&\overset{\mathcal{Z}}{ \longleftrightarrow}X\brak{z^k}
    \end{align}
    \item Time reversal property:    \\
    \begin{align}
        x\brak{-n}&\overset{\mathcal{Z}}{ \longleftrightarrow}X\brak{z^{-1}}
    \end{align}
    \item $Z-$domain scaling:   \\
    \begin{align}
        a^{n}x\brak{n}&\overset{\mathcal{Z}}{ \longleftrightarrow}X\brak{\frac{z}{a}}
    \end{align}
    \item Convolution property: \\
    \begin{align}
        x\brak{n}\ast y\brak{n}&\overset{\mathcal{Z}}{ \longleftrightarrow}X\brak{z}Y\brak{z}
    \end{align}
    \item Differentiation in $Z-$Domain:
    \begin{align}
        n^{k}x\brak{n}&\overset{\mathcal{Z}}{ \longleftrightarrow}\brak{-1}^{k}z^{k}\frac{d^{k}X\brak{z}}{dz^{k}}
    \end{align}
\end{enumerate}
\end{document}
