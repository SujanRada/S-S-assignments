% \iffalse
\let\negmedspace\undefined
\let\negthickspace\undefined
\documentclass[journal,12pt,twocolumn]{IEEEtran}
\usepackage{cite}
\usepackage{amsmath,amssymb,amsfonts,amsthm}
\usepackage{algorithmic}
\usepackage{graphicx}
\usepackage{textcomp}
\usepackage{xcolor}
\usepackage{pgfplots}
\usepackage{txfonts}
\usepackage{listings}
\usepackage{enumitem}
\usepackage{mathtools}
\usepackage{gensymb}
\usepackage{comment}
\usepackage[breaklinks=true]{hyperref}
\usepackage{tkz-euclide} 
\usepackage{listings}
\usepackage{gvv}                                        
\def\inputGnumericTable{}                                 
\usepackage[latin1]{inputenc}                                
\usepackage{color}                                            
\usepackage{array}                                            
\usepackage{longtable}                                       
\usepackage{calc}                                             
\usepackage{multirow}                                         
\usepackage{hhline}                                           
\usepackage{ifthen}                                           
\usepackage{lscape}

\newtheorem{theorem}{Theorem}[section]
\newtheorem{problem}{Problem}
\newtheorem{proposition}{Proposition}[section]
\newtheorem{lemma}{Lemma}[section]
\newtheorem{corollary}[theorem]{Corollary}
\newtheorem{example}{Example}[section]
\newtheorem{definition}[problem]{Definition}
\newcommand{\BEQA}{\begin{eqnarray}}
\newcommand{\EEQA}{\end{eqnarray}}
\newcommand{\define}{\stackrel{\triangle}{=}}
\theoremstyle{remark}
\newtheorem{rem}{Remark}
\begin{document}
\parindent 0px
\bibliographystyle{IEEEtran}
\title{GATE: ME - 14.2022}
\author{EE22BTECH11219 - Rada Sai Sujan$^{}$% <-this % stops a space
}
\maketitle
\newpage
\bigskip
\section*{Appendix}
\subsection*{Complex Fourier series:}
\begin{align}
	x(t) &= \sum_{n=-\infty}^{\infty} c_{n}e^{j2\pi nft}
	\label{eq:compfs}
\end{align}
where $c_{n}$ is the exponential fourier coefficient.
\begin{align}
        c_{n} &= \frac{1}{T} \int_{0}^{T} x(t) e^{-j2\pi nft} \, dt
\end{align}
where $T$ is the time period of function $x(t)$.    \\
\subsection*{Trignometric fourier series:}
\begin{align}
	e^{j2\pi nft} = cos(2\pi nf t) + j sin(2\pi nft)
	\label{eq:expfs}
\end{align}
Substituting \eqref{eq:expfs} in \eqref{eq:compfs}
\begin{align}
	 x(t) &= \sum_{n=-\infty}^{\infty} c_{n} \brak{\cos(2\pi nft) + j \sin(2\pi nft)}\\
	      &= a_0 + \sum_{n=1}^{\infty} \brak{a_{n} \cos(2\pi nft)} + \brak{b_{n} \sin(2\pi nft)}
\end{align}
where $a_0, a_{n}$ and $b_{n}$ are trigonometric fourier series.
\begin{align}
          a_{0} &= c_{0}\\
                &=\frac{1}{T} \int_{0}^{T} x(t) \, dt\\
          a_{n} &= 2Re(c_{n})\\
                &=\frac{2}{T} \int_{0}^{T} x(t)\cos(2\pi nft) \, dt\\
          b_{n} &= -2Im(c_{n})\\
                &= \frac{2}{T} \int_{0}^{T} x(t)\sin(2\pi nft) \, dt
\end{align}
Therefore, Fourier series expansion of the function $x\brak{t}$ in the interval $[-L,L]$ can be given by: \\
\begin{align}
    x\brak{t}=a_0+\sum\limits_{n=1}^{\infty}a_n\cos\brak{\frac{n\pi t}{L}}+\sum\limits_{n=1}^{\infty}b_n\sin\brak{\frac{n\pi t}{L}}
\end{align}
where,
\begin{align}
    f&=\frac{1}{2L}	\\
    a_0&=\frac{1}{2L}\int\limits_{-L}^{L}f\brak{t}\,dt  \\
    a_n&=\frac{1}{2L}\int\limits_{-L}^{L}f\brak{t}\cos\brak{\frac{n\pi t}{L}}\,dt  \\
    b_n&=\frac{1}{2L}\int\limits_{-L}^{L}f\brak{t}\sin\brak{\frac{n\pi t}{L}}\,dt  \\
\end{align}
\end{document}
