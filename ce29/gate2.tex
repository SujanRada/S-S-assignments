% \iffalse
\let\negmedspace\undefined
\let\negthickspace\undefined
\documentclass[journal,12pt,twocolumn]{IEEEtran}
\usepackage{cite}
\usepackage{amsmath,amssymb,amsfonts,amsthm}
\usepackage{algorithmic}
\usepackage{graphicx}
\usepackage{textcomp}
\usepackage{xcolor}
\usepackage{txfonts}
\usepackage{listings}
\usepackage{enumitem}
\usepackage{mathtools}
\usepackage{gensymb}
\usepackage{comment}
\usepackage[breaklinks=true]{hyperref}
\usepackage{tkz-euclide} 
\usepackage{listings}
\usepackage{gvv}                                        
\def\inputGnumericTable{}                                 
\usepackage[latin1]{inputenc}                                
\usepackage{color}                                            
\usepackage{array}                                            
\usepackage{longtable}                                       
\usepackage{calc}                                             
\usepackage{multirow}                                         
\usepackage{hhline}                                           
\usepackage{ifthen}                                           
\usepackage{lscape}

\newtheorem{theorem}{Theorem}[section]
\newtheorem{problem}{Problem}
\newtheorem{proposition}{Proposition}[section]
\newtheorem{lemma}{Lemma}[section]
\newtheorem{corollary}[theorem]{Corollary}
\newtheorem{example}{Example}[section]
\newtheorem{definition}[problem]{Definition}
\newcommand{\BEQA}{\begin{eqnarray}}
\newcommand{\EEQA}{\end{eqnarray}}
\newcommand{\define}{\stackrel{\triangle}{=}}
\theoremstyle{remark}
\newtheorem{rem}{Remark}
\begin{document}
\parindent 0px
\bibliographystyle{IEEEtran}
\title{GATE: CE - 29.2022}
\author{EE22BTECH11219 - Rada Sai Sujan$^{}$% <-this % stops a space
}
\maketitle
\newpage
\bigskip
\section*{Question}
Consider the following recursive iteration scheme for different values of variable P with the initial guess $x_1=1$:
$$x_{n+1}=\frac{1}{2}\brak{x_n+\frac{P}{x_n}}, \quad\quad n=1,2,3,4,5 $$
For $P=2$, $x_5$ is obtained to be 1.414, rounded off to 3 decimal places. For $P=3$, $x_5$ is obtained to be 1.732, rounded off to 3 decimal places.   \\ \\
If $P=10$, the numerical value of $x_5$ is \rule{1.3cm}{0.15mm} . ($round$ $off$ $to$ $three$ $decimal$ $places$)   \\
\solution 

Applying $A.M \geq G.M$ inequality,
\begin{align}
    \frac{x_n+\frac{P}{x_n}}{2} \geq \sqrt{P}   \\
    \implies x_{n+1} \geq \sqrt{P}  \label{eq:ce.29.22.1}
\end{align}
Solving the equation,
\begin{align}
    2x_{n+1}x_{n} - {x_n}^2 - P &=0
\end{align}
Applying $Z$-transform we get,
\begin{align}
    X\brak{z} \ast X\brak{z} &= \frac{PZ^{-1}}{\brak{1-z^{-1}}\brak{2-z^{-1}}}  \\
    &= P\brak{\frac{z^{-1}}{1-z^{-1}} - \frac{z^{-1}}{2-z^{-1}}}
\end{align}
Now, applying inverse $Z$-tranform,
\begin{align}
    x_n^2 &= P\brak{u\brak{n-1} - \frac{u\brak{n-1}}{2^n}}  \\
    \implies x_n^2 &=P\brak{1-\frac{1}{2^n}} \quad \text{[$\because n \geq 1$]}
\end{align}
Similarly,
\begin{align}
    x_{n+1}^2 &= P\brak{1 - \frac{1}{2^{n+1}}}  \\
    \implies x_{n+1}^2 - x_n^2 &= \frac{P}{2^{n+1}}\quad \brak{<0}
\end{align}
Hence, the system is convergent as it is increasing and bounded,    \\
Now finding the limit of the sequence,
\begin{align}
    x^2 &=\lim\limits_{x\to\infty}P\brak{1-\frac{1}{2^n}}   \\
    \implies x &= \pm\sqrt{P}   \label{eq:ce.29.22.11}
\end{align}
From \eqref{eq:ce.29.22.1} and \eqref{eq:ce.29.22.11},
\begin{align}
    x_n=\sqrt{P}
\end{align}
Therefore, for $P=10$ the value of $x_5$ is,
\begin{align}
    x_5&=\sqrt{10}  \\
    \therefore x_5&=3.162
\end{align}
\end{document}
